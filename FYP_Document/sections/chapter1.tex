\chapter{Introduction}
\label{sec:introduction}

The purpose of this project proposal is to define and present the scope, objectives, and methodology for the development of \textbf{Zaiqa: A Context-Aware Cooking Assistant for Home Kitchens}. This document provides a detailed background of the system, highlights the problems faced by home cooks, and outlines how the proposed software solution will address those challenges. The system aims to offer a culturally relevant and interactive approach to food recommendation by leveraging artificial intelligence techniques. The proposal will also define the problem statement, motivation, objectives, and major stakeholders involved in the project. \cite{test}

\section{Problem Statement}

Meal preparation and food decision-making remain significant challenges for many households, especially in Pakistan, where food choices are often influenced by cultural preferences, family needs, and budget constraints. Most individuals struggle daily with deciding what to cook, often due to limited knowledge of recipes or a lack of inspiration when restricted by the ingredients available at home. Current recipe platforms and food blogs provide generic lists of dishes, but they do not offer personalized recommendations based on budget, age suitability, or ingredient availability. Furthermore, these platforms lack support for Roman Urdu or bilingual communication, which makes them less accessible for local users. This results in wasted time, repeated meals, and a less enjoyable cooking experience. In addition, there is no interactive cooking assistant that can guide users step-by-step in an engaging and user-friendly manner. As a result, home cooks, especially beginners, find it difficult to experiment with new dishes or manage healthy and cost-effective meal planning. The absence of feedback-driven personalization further reduces the usefulness of existing platforms, as they fail to adapt to the preferences of individual households. Therefore, there is a need for a system that bridges these gaps by combining natural language processing, recommendation engines, and interactive cooking support into one cohesive platform. The proposed system, Zaiqa, aims to fill this gap by providing personalized, context-aware, and culturally relevant recipe suggestions tailored specifically to Pakistani households.

\section{Motivation}

The primary motivation for selecting this problem is the increasing demand for intelligent systems that assist individuals in managing everyday household tasks. Cooking is an essential part of daily life, yet it remains a source of stress for many due to lack of time, budget limitations, or limited knowledge of recipe variations. With the rise of AI-driven recommendation systems in entertainment, shopping, and fitness, a similar solution tailored for cooking can greatly simplify meal planning. Existing food delivery platforms and recipe blogs are not sufficient, as they ignore cultural and linguistic aspects unique to Pakistan. Furthermore, the younger generation often relies on digital platforms for quick solutions, making an AI-based home cooking assistant both timely and practical. Addressing this gap will not only make meal planning easier but also help promote healthy eating habits, reduce food waste by utilizing available ingredients, and encourage diversity in meals at home. This project also provides us as students with an opportunity to apply concepts of natural language processing, recommendation systems, and human-computer interaction in a real-world context.

\section{Problem Solution}

To address the challenges identified in the problem statement, the proposed solution, Zaiqa, will act as a culturally aware, AI-powered cooking assistant tailored for Pakistani households. The application will allow users to input cravings or available ingredients in Roman Urdu or English and will recommend suitable recipes based on budget, age group, and meal type. Additionally, it will provide interactive, step-by-step cooking instructions that make meal preparation simple and enjoyable. Unlike generic platforms, Zaiqa will integrate feedback mechanisms to adapt to user preferences over time. The overall goal of this project is to reduce the stress of daily meal decision-making while promoting variety, affordability, and healthier eating habits.  

\textbf{Objectives:}  
\begin{itemize}
    \item Develop a multilingual craving understanding engine for Roman Urdu and English.
    \item Build a recipe recommendation engine based on available ingredients, budget, and age suitability.
    \item Provide a step-by-step interactive cooking assistant for guided meal preparation.
    \item Create a dataset of Pakistani recipes annotated with cooking time, cost, and age group.
    \item Implement a feedback-driven personalization loop to refine recommendations.
    \item Ensure a user-friendly interface for both mobile and web platforms.
    \item (Optional) Integrate allergy-based filtering for safe recipe suggestions.
\end{itemize}

In summary, the system will not only simplify the cooking decision process but also enhance the overall cooking experience through personalization, cultural relevance, and interactive guidance.

\section{Stake Holders}

The key stakeholders for the project are:  
\begin{itemize}
    \item \textbf{Primary Users:} Home cooks, families, and individuals who regularly prepare meals at home.  
    \item \textbf{Secondary Users:} Beginners or young adults learning to cook and seeking guidance.  
    \item \textbf{Project Team:} Developers (students) responsible for designing, developing, and testing the system.  
    \item \textbf{Supervisors and Instructors:} Academic mentors providing guidance and evaluating progress.  
    \item \textbf{Dataset Providers:} Online recipe platforms such as Masala.tv and Food Fusion, serving as sources of data.  
    \item \textbf{Future Stakeholders:} Nutritionists, food bloggers, or local culinary experts who may contribute insights for enhancement.  
\end{itemize}

